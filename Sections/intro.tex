\TeX{} is a low-level programming language created by Donald Knuth in 1977 to typeset documents. The letters of the name represent the
capital Greek letters tau, epsilon, and chi, as \TeX{} is an abbreviation of the Greek word for both ``art" and ``craft". It is therefore pronounced as the fist syllable of ``technical". However, in Afrikaans it's pronounced as the first syllable of ``tegnies".

To simplify the typesetting, Leslie Lamport then developed a macro package called \LaTeX{}, pronounced as ``Lay-tech" or ``Lah-teg". Many later authors have contributed extensions, called \textit{packages} or \textit{styles}, to \LaTeX{}.

Since \LaTeX{} word processing is essentially programming, some users may shy away from the steep learning curve. However, since the formatting is applied consistently throughout the document by commands, it offers an advantage over WYSIWYG (what you see is what you get) word processors like Microsoft Word. The focus thus shifts from superficial formatting issues to writing good content.

A number of online resources that you will definitely need to consult as you continue to use \LaTeX{} are listed below.
The quickest solution is simply to search your question in your favourite search engine with inclusion of the word `latex'.

\begin{itemize}
    \item \href{https://wch.github.io/latexsheet/}{\LaTeX{} Cheat Sheet}
    \item \href{https://tobi.oetiker.ch/lshort/lshort.pdf}{The Not So Short Introduction to \LaTeXe}
    \item \href{https://en.m.wikibooks.org/wiki/LaTeX}{\LaTeX{} wikibook}
    \item \href{http://ftp.leg.uct.ac.za/pub/packages/ctan/info/short-math-guide/short-math-guide.pdf}{Short Math Guide for \LaTeX}
    \item \href{http://tex.stackexchange.com}{\TeX{} Stack Exchange}
\end{itemize}